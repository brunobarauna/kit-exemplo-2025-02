% Documento LaTeX com o arquivo que estamos escrevendo

% Cabeçalho
% Onde a gente configura o documento
\documentclass{article}
\usepackage[brazil]{babel}


% Corpo
% Onde a gente escreve o texto
%%%%%%%%%%%%%%%%%%%%%%%%%%%%%%%%%%%%%%%%%%%%%%%%%%%%%%%%%%%%%%%%%%%%

\begin{document}
\title{Análise de variação de temperatura dos últimos cinco anos}
\author{Bruno Barauna}
\maketitle

\begin{abstract}
Meu resumo legalzão
\end{abstract}

\section{Introdução}

Escola de verão da USP.
outra frase aqui

latex

\section{Metodologia}
\label{sec:metodos}
Aqui vou descrever tudo que fizemos.
Ajustamos uma reta aos cinco últimos anos dos dados
de temperatura média mensal para cada país.
Assim calculamos a taxa de variação de temperatura recente.

A equação da reta é

\begin{equation}
T(t) = a t +b,
\label{eq:reta}
\end{equation}

\noindent
onde $T$ é a temperatura, $t$ é o tempo, $a$ é o coeficiente angular e $b$ é o coeficiente linear.

Ultilizamos a equação \ref{reta} em um código Python para fazer o ajuste da 
reta com mínimos quadrados.
Isso está decrito na seção \ref{sec:metodos}.

\end{document}