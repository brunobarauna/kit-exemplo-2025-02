% Documento LaTeX com o arquivo que estamos escrevendo

% Cabeçalho
% Onde a gente configura o documento
\documentclass{article}
\usepackage[brazil]{babel}
\usepackage{graphicx}
\usepackage[round,authoryear,sort]{natbib}

% Corpo
% Onde a gente escreve o texto
%%%%%%%%%%%%%%%%%%%%%%%%%%%%%%%%%%%%%%%%%%%%%%%%%%%%%%%%%%%%%%%%%%%%

\begin{document}
\title{Análise de variação de temperatura dos últimos cinco anos}
\author{Bruno Barauna}
\maketitle

\begin{abstract}
Meu resumo legalzão
\end{abstract}

\section{Introdução}

Escola de verão da USP.
outra frase aqui

Trabalhos anteriores bem legal que fizeram coisas parecidas
\citep{Hansen2010}.

Isso foi analisado primeiro por \citet{Hansen2010}.

\section{Metodologia}
\label{sec:metodos}
Aqui vou descrever tudo que fizemos.
Ajustamos uma reta aos cinco últimos anos dos dados
de temperatura média mensal para cada país.
Assim calculamos a taxa de variação de temperatura recente.

A equação da reta é

\begin{equation}
T(t) = a t +b,
\label{eq:reta}
\end{equation}

\noindent
onde $T$ é a temperatura, $t$ é o tempo, $a$ é o coeficiente angular e $b$ é o coeficiente linear.

Ultilizamos a equação \ref{reta} em um código Python para fazer o ajuste da 
reta com mínimos quadrados.
Isso está decrito na seção \ref{sec:metodos}.

\section{Resultados}
\label{sec:resultados}

Analisamos os dados de 225 países. São muitos para listar aqui. Confia.
%! - muito por favor
\begin{figure}[!htb]
	\centering
	\includegraphics[width=0.5\columnwidth]{../figuras/variacao_temperatura.png}
	\caption{
		Variação de temperatura média mensal dos últimos cinco anos.
		a) Países com as cinco maiores variações de temperatura.
		b) Países com as cinco menores variações de temperatura.
	}
	\label{fig:variacoes}
\end{figure}

Os resultados da análise de variação de temperatura estão na figura \ref{fig:variacoes}.

\bibliographystyle{apalike}
\bibliography{referencias.bib}
\end{document}